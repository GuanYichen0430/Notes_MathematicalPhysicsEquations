% !TEX root = ../mat999.tex
\newpage
\section{线性偏微分方程}
\subsection{常微分方程的情形}
\subsubsection{常系数齐次ODE}
针对二阶线性齐次方程$x''+a_1x'+a_2x=0, \lambda_1,\lambda_2$为其特征方程$\lambda^2+a_1\lambda+a_2=0$的两根
$(\lambda_{1,2}=\alpha\pm i\beta)$,其通解为:

$$\begin{aligned}
    (i)&&\Delta>0:\quad&x=c_1e^{\lambda_1t}+c_2e^{\lambda_2t}\\
    (ii)&&\Delta=0:\quad&x=(c_1+c_2t)e^{\lambda_1t}\\
    (iii)&&\Delta<0:\quad&x=(c_1\cos\beta t+c_2\sin\beta t)e^{\alpha t}
\end{aligned}$$

\subsubsection{非齐次ODE}
$$\left\{
    \begin{aligned}
    &
    y''(x)+P(x)y'(x)+Q(x)y(x)=F(x)\\
    &y|_{x=a}=0,y|_{x=b}=0
            \end{aligned}
    \right.$$

    关键是先写出一个特解$y_p(x)$满足ODE,不必满足边界条件
    
    通解$y(x)=C_1y_1(x)+C_2y_2(x)+y_p(x)$

    $y_1(x),y_2(x)$为齐次方程的一组线性无关解:
$$y_1''+Py_1'+Qy_1=0,y_2''+Py_2'+Qy_2=0$$

$y_p(x)=\int_a^xw(x;s)ds$是非齐次ODE的特解,可由齐次化原理写出

\subsection{线性偏微分方程的一般理论:叠加性和解的结构}
把线性偏微分方程统一写成算符形式:
$$L(u)=f$$
$$\begin{aligned}
    \text{波动方程:}&&\frac {\partial ^2u}{\partial t^2}- a^2\frac {\partial ^2u}{\partial x^2}=f
    &\quad L=\frac{\partial^2}{\partial t^2}-a^2\frac{\partial^2}{\partial x^2}\\
    \text{热传导:}&&\frac{\partial{u}}{\partial{t}}-D\nabla^2u=f
    &\quad L=\frac\partial{\partial t}-D\nabla^2\\
    \text{泊松方程:}&&\nabla^2u=f
    &\quad L=\nabla^2
\end{aligned}$$
其中

u:未知函数

f:已知函数,方程的非齐次项

L:线性算符

$f=0$:齐次方程

$f\neq0$:非齐次方程
\begin{thm}[解的叠加原理]$\ $

    1. $L(u_1)=0,L(u_2)=0\Rightarrow L(c_1u_1+c_2u_2)=0$

    2. $L(u_1)=0,L(u_2)=f\Rightarrow L(u_1+u_2)=f$

    非齐次方程的特解+相应齐次方程的解仍然是非齐次方程的解

    非齐次方程的通解 =非齐次方程的任一特解 + 相应齐次方程的通解

    3. $L(u_1)=f_1,L(u_2)=f_2\Rightarrow L(c_1u_1+c_2u_2)=c_1f_1+c_2f_2$

    4. $u_1,u_2,...u_n: L(u_1)=f_1, L(u_2)=f_2,...,L(u_n)=f_n$
    
    $\Rightarrow L(c_1u_1+c_2u_2+...+c_nu_n)=c_1f_1+c_2f_2+...+c_nf_n$
\end{thm}


\subsection{齐次问题:波动方程的行波解}
\subsubsection{无界问题}
对于无界的波动方程:
$$\frac{\partial^2{u}}{\partial{t}^2}-a^2\frac{\partial^2{u}}{\partial{x}^2}=0, -\infty<x<\infty, t>0$$ 
通解:
$$\boxed{u(x,t)=f(x-at)+g(x+at)}$$
若给定初始条件和边界条件$u|_{t=0}=\phi(x),\frac{\partial u}{\partial t}|_{t=0}=\psi(x)$,有
$$u(x,t)=\frac{1}{2}[\phi(x-at)+\phi(x+at)]+\frac{1}{2a}\int^{x+at}_{x-at}\psi(\xi)d\xi$$
其中,
$$\mbox{右行波:}f(x-at)=\frac{1}{2}\phi(x-at)-\frac{1}{2a}\int^{x-at}_{0}\psi(\xi)d\xi$$
$$\mbox{左行波:}g(x+at)=\frac{1}{2}\phi(x+at)+\frac{1}{2a}\int^{x+at}_{0}\psi(\xi)d\xi$$

\subsubsection{半无界问题}
$$\left\{
    \begin{aligned}
            &\frac{\partial^2{u}}{\partial{t}^2}-a^2\frac{\partial^2{u}}{\partial{x}^2}=0,x>0, t>0\\
&u|_{x=0}=0,\\
            &u|_{t=0}=\phi(x),\frac{\partial u}{\partial t}\bigg|_{t=0}=\psi(x)
    \end{aligned}
\right.$$

做\textbf{奇延拓}:定义
$$\Phi=\left\{
    \begin{aligned}
        \phi(x),&x>0\\
        -\phi(-x),&x<0
    \end{aligned}
\right.,\quad
\Psi=\left\{
        \begin{aligned}
            \psi(x),&x>0\\
            -\psi(-x),&x<0
        \end{aligned}
\right.$$

$$\begin{aligned}
    u(x,t)=&\frac{1}{2}[\Phi(x-at)+\Phi(x+at)]+\frac{1}{2a}\int^{x+at}_{x-at}\Psi(\xi)d\xi\\
    =&\left\{
    \begin{aligned}
        \frac{1}{2}[\phi(x-at)+\phi(x+at)]+\frac{1}{2a}\int^{x+at}_{x-at}\psi(\xi)d\xi,\quad&t\le\frac{x}{2}\\
        \frac{1}{2}[-\phi(at-x)+\phi(x+at)]+\frac{1}{2a}\int^{x+at}_{at-x}\psi(\xi)d\xi,\quad&t>\frac{x}{2}
    \end{aligned}
    \right.
    \end{aligned}$$

\subsection{非齐次方程的齐次化原理}
\begin{ex}[波动方程(对二阶的PDE齐次化原理)]
$$      \left\{
        \begin{aligned}
                &\frac{\partial^2{u}}{\partial{t}^2}=a^2\frac{\partial^2{u}}{\partial{x}^2}+f(x,t), -\infty<x<\infty, t>0\\
                &u|_{t=0}=0,\frac{\partial u}{\partial t}\bigg|_{t=0}=0
        \end{aligned}
\right.$$
\begin{rem}
    若初始条件是$u|_{t=0}=\phi(x),\frac{\partial u}{\partial t}\bigg|_{t=0}=\psi(x)$,只需将问题拆解成(1), (2)两部分求解$u=u_1+u_2$
    $$      (1)\left\{
        \begin{aligned}
                &\frac{\partial^2{u_1}}{\partial{t}^2}=a^2\frac{\partial^2{u_1}}{\partial{x}^2}\\
                &u_1|_{t=0}=\phi(x),\frac{\partial u_1}{\partial t}\bigg|_{t=0}=\psi(x)
        \end{aligned}
\right.\quad (2)\left\{
        \begin{aligned}
                &\frac{\partial^2{u_2}}{\partial{t}^2}=a^2\frac{\partial^2{u_2}}{\partial{x}^2}+f\\
                &u_2|_{t=0}=0,\frac{\partial u_2}{\partial t}\bigg|_{t=0}=0
        \end{aligned}
\right.$$
\end{rem}

\noindent\textbf{Step 1 分割} 

在$[0,t_0]$区间里解方程:将$[0,t_0]$分成$N$份
$$N\Delta t=t_0$$

$f_n$为脉冲力,仅在小区间内非零
$$f_1(x,t)+f_2(x,t)+...+f_N(x,t)=f(x,t)$$
$$      \left\{
        \begin{aligned}
                &\frac{\partial^2{u_n}}{\partial{t}^2}-a^2\frac{\partial^2{u_n}}{\partial{x}^2}=f(x,t), -\infty<x<\infty, t>0\\
                &u_n|_{t=0}=\frac{\partial u_n}{\partial t}\bigg|_{t=0}=0
        \end{aligned}
\right.\quad\Rightarrow u(x,t)=u_1(x,t)+...+u_N(x,t)$$

\noindent\textbf{Step 2 求解脉冲力问题} 
$$\tau=n\Delta t$$
$$f_n|_{t<\tau-\Delta t}=0 \\\Rightarrow u_n|_{t<\tau-\Delta t}=\frac{\partial u_n}{\partial t}\bigg|_{t=0}=0$$

在小区间内,$f_n$近似为常数:$f_n(x,t)\approx f(x,\tau) t\in[\tau-\Delta t,\tau]$

经过区间$[\tau-\Delta t,\tau]$,外力$f$产生速度$\frac{\partial u_n}{\partial t}\bigg|_{t=\tau}=f(x,t)\Delta t$,位移$u_n|_{t=\tau}\sim(\Delta t)^2$可忽略:
  $$\frac{\partial^2{u_n(x,t)}}{\partial{t}^2}=f(x,\tau)\Rightarrow\frac{\partial u_n}{\partial t}\bigg|_{t=\tau}-\frac{\partial u_n}{\partial t}\bigg|_{t=\tau-\Delta t}=f(x,\tau)\Delta t$$
  
  ($[\tau-\Delta t,\tau]$内,$a^2\frac{\partial^2{u_n}}{\partial{x}^2}\sim(\Delta t)^2$,可忽略)
$$t>\tau:\left\{
        \begin{aligned}
                &\frac{\partial^2{u_n}}{\partial{t}^2}-a^2\frac{\partial^2{u_n}}{\partial{x}^2}=0\\
                &u_n|_{t=\tau}=0,\frac{\partial u_n}{\partial t}\bigg|_{t=\tau}=f(x,\tau)\Delta t
        \end{aligned}
\right.$$

$$\text{定义:}w(x,t;\tau_n)=\frac{u_n(x,t)}{\Delta t}\Rightarrow
\left\{
        \begin{aligned}
                &\frac{\partial^2{w}}{\partial{t}^2}-a^2\frac{\partial^2{w}}{\partial{x}^2}=0\\
                &w|_{t=\tau}=0,\frac{\partial w}{\partial t}\bigg|_{t=\tau}=f(x,\tau)
        \end{aligned}
\right.$$
\noindent\textbf{Step 3 $\Delta t\rightarrow0$,积分} 
$$\begin{aligned}
u(x,t)=&u_1(x,t)+u_2(x,t)+...+u_N(x,t)\\
=&w(x,t;\tau_1)\Delta t+w(x,t;\tau_2)\Delta t+...+w(x,t;\tau_N)\Delta t\\
\xlongequal{\Delta t\rightarrow0}&\int_0^t{w(x,t;\tau)dt}
\end{aligned}$$
根据行波解:$$w(x,t;\tau)=\frac{1}{2a}\int^{x+a(t-\tau)}_{x-a(t-\tau)}f(\xi,\tau)d\xi$$
\end{ex}

\begin{mtd}
$$u(x,t)=\int^t_0{w(x,t;\tau)d\tau}$$
$$w(x,t;\tau)\text{满足}\left\{
        \begin{aligned}
                &\frac{\partial^2{w}}{\partial{t}^2}-a^2\frac{\partial^2{w}}{\partial{x}^2}=0,t>\tau\\
                &w(x,t;\tau)|_{t=\tau}=0,\frac{\partial w}{\partial t}\bigg|_{t=\tau}=f(x,\tau)
        \end{aligned}
\right.$$
$$w(x,t;\tau)=\frac{1}{2a}\int^{x+a(t-\tau)}_{x-a(t-\tau)}f(\xi,\tau)d\xi$$
\end{mtd}

