\newpage
\section{Sturm-Liouville 理论}
\subsection{S-L问题}

\noindent 应对本征值问题的复杂性
\begin{enumerate}
    \item S-L理论:直接导出解的性质
    \item 级数解法
\end{enumerate}

\begin{dfn}[S-L型方程的一般形式]
    $$\frac{d}{dx}\left[p(x)\frac{dy(x)}{dx}\right]-q(x)y(x)+\lambda\rho(x)y(x)=0\quad,0<x<b$$
    引进算符$$L=-\frac{d}{dx}(p(x)\frac{d}{dx})+q(x)$$
    方程可以写成:$$Ly(x)=\lambda\rho(x)y(x)$$
    加适当的齐次边界条件
\end{dfn}

$p(x),q(x),\rho(x)$非常数:非均匀性的体现(材料不均匀/曲线坐标系)

考虑$p(x),q(x),\rho(x)$为实值函数,连续函数
在开区间$(a,b)$上
$$\begin{cases}
p(x)>0\\
q(x)\ge 0\\
\rho(x)>0
\end{cases}$$

\subsection{自伴算符的本征值问题}
\subsubsection{自伴算符}
\begin{dfn}[自伴性]
    $$\langle L f,g\rangle=\langle f,Lg\rangle$$
    则称$L$具有自伴性
    $L$的伴算子$L^\dagger$定义为:
    $$\langle L^\dagger f,g\rangle=\langle f,Lg\rangle$$
    自伴:$$L^\dagger=L$$
\begin{rem}
    函数内积定义$$\langle f,g\rangle=\int_a^bf^*(x)g(x)dx$$
\end{rem}
\end{dfn}
\subsubsection{L算符的自伴性}
$$(Lf)^*g-f^*(Lg)=\frac{d}{dx}\left[p(x)\left(f^*\frac{dg}{dx}-g\frac{df^*}{dx}\right)\right]$$
两边积分:
$$\int_a^b[(Lf)^*g-f^*(Lg)]dx=p(x)\left[f^*\frac{dg}{dx}-g\frac{df^*}{dx}\right]\bigg|_a^b$$
$$\langle Lf,g\rangle-\langle f,Lg\rangle=\text{边界项}$$
若边界项$=0\Rightarrow\langle Lf,g\rangle=\langle f,Lg\rangle$,L具有自伴性

自伴性强烈依赖边界条件,只有当边界条件使得边界项$=0$时,$L$具有自伴性

\noindent\textbf{边界项$=0$的几种典型实现方式}
$$p(x)\left[f^*\frac{dg}{dx}-g\frac{df^*}{dx}\right]\bigg|_a^b=0$$
\noindent 1. 每个端点(a,b)处,加第一、二、三类边界条件,使两边界各自为零

第一类:$$\begin{cases}
f(a)=g(a)=0\\
f(b)=g(b)=0
\end{cases}$$

第二类:$$\begin{cases}
f'(a)=g'(a)=0\\
f'(b)=g'(b)=0
\end{cases}$$

第三类:$$\begin{cases}
f'(a)=\alpha f(a),g'(a)=\alpha g(a)\\
f'(b)=\beta f(b),g'(b)=\beta g(b)
\end{cases}$$

\noindent 2. 两端点边界项抵消
$$p(a)=p(b)$$

周期边界条件:$$\begin{cases}
f(a)=f(b),f'(a)=f'(b)\\
g(a)=g(b),g'(a)=g'(b)
\end{cases}$$

\noindent 3. 端点$p(x)=0$,加有界条件
  $$p(a)=p(b)=0, y(a),y(b),y'(a),y'(b)<\infty$$

\subsubsection{自伴算符的基本性质}

\noindent\textbf{1. 本征值的可数性}:自伴算符的本征值必然存在。本征值有无穷多个,构成可数集
$$\lambda_1\le\lambda_2\le...\le\lambda_n\le...$$
$$\lim_{n\rightarrow\infty}\lambda_n=+\infty$$

\noindent\textbf{2. 本征值的实数性}$\lambda=\lambda^*$

\noindent\textbf{3. 本征函数的正交性}:对应不同本征值的本征函数一定正交$$\int_a^b\rho(x)f^*(x)g(x)dx=0$$

\noindent\textbf{4. 本征值的非负性}$\lambda\ge0$

若$\rho(a)f^*(a)f'(a)-\rho(b)f^*(b)f'(b)\ge0$,则$\lambda\ge0$

第一、二类边界条件显然满足

第三类边界条件
$$\begin{cases}
\alpha_1f(a)+\beta_1f'(a)=0 &\Rightarrow f'(a)=-\frac{\alpha_1}{\beta_1}f(a)\\
\alpha_2f(b)+\beta_2f'(b)=0 &\Rightarrow f'(b)=-\frac{\alpha_2}{\beta_2}f(b)
\end{cases}$$
$$\alpha_1\beta_1<0,\alpha_2\beta_2>0\Rightarrow\lambda\ge0$$

排除非物理情形,第三类边界条件也有$\lambda\ge0$

\noindent\textbf{5. 完备性}

自伴算符的本征函数(的全体)构成一个完备的函数组,即任意一个在区间$[a,b]$中有连续二阶导数、且满足和自伴算符$L$相同的边界条件的函数$f(x)$,均可按本征函数$\{y_n(x)\}$展开为绝对而且一致收敛的级数。
$$f(x)=\sum_{n=1}^\infty C_nX_n(x)$$
$$C_n=\frac{\int_a^b\rho(x)X_n^*f(x)}{||X_n||^2}=\frac{\int_a^b\rho(x)X_n^*f(x)}{\int_a^b\rho(x)|X_n(x)|^2dx}$$
$$||X_n||^2=\int_a^b\rho(x)|X_n(x)|^2dx$$

\subsection{S-L型方程的本征值问题}
将方程化为S-L方程的标准形式
$$\frac{d}{dx}\left[p(x)\frac{dy(x)}{dx}\right]-q(x)y(x)+\lambda\rho(x)y(x)=0$$

一般方程:
$$y''(x)+a(x)y'(x)+b(x)y(x)+\lambda c(x)y(x)=0$$
$$e^{\int^x a(x')dx'}[y''+ay'+by+\lambda cy]=0$$
$$\frac{d}{dx}[e^{\int^x a(x')dx'}\frac{dy}{dx}]+b(x)e^{\int^x a(x')dx'}y+\lambda c(x)e^{\int^x a(x')dx'}y=0$$
$$p(x)=e^{\int^x a(x')dx'}$$
$$q(x)=-b(x)e^{\int^x a(x')dx'}$$
$$\rho(x)=c(x)e^{\int^x a(x')dx'}$$

